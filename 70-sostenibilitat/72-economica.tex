\section{Dimensió econòmica}

\subsubsection{Has estimat el cost de la realització del projecte (recursos humans i materials)?}
Sí, s'ha fet una anàlisi de la gestió econòmica del projecte on es tracta, tant el cost de la realització del projecte, com les seves parts individuals i també s'ha tingut en compte el control de gestió per evitar inversions innecessàries.

\subsubsection{Com es resol actualment el problema que vols tractar (estat de l'art)? En què millora econòmicament la teva solució a les ja existents?}
El mètode actual de resolució d'incidències mitjançant sistemes de tiquets depèn en gran manera de processos manuals d'anàlisi i de resolució, que sovint exigeixen grans quantitats de temps i recursos humans. No obstant això, aquest mètode és intrínsecament limitat a causa de la seva naturalesa reactiva i de la seva incapacitat per aprofitar eficaçment la gran quantitat de dades textuals que contenen els tiquets. La solució que es proposa permet extreure informació valuosa de les incidències. L'automatització de l'anàlisi de les dades permet la identificació proactiva de possibles amenaces. Això agilitza la resolució d'incidències i optimitza la utilització de recursos en abordar els problemes recurrents de manera preventiva.
