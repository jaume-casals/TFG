\section{Autoavaluació}
És fonamental reflexionar sobre les pròpies conclusions sobre la sostenibilitat i el desenvolupament sostenible abans d'embarcar-se en aquest projecte, especialment en el context de l'àmbit informàtic. Això s'ha aconseguit responent al qüestionari del projecte \textbf{EDINSOST2-ODS}.

La integració de coneixements generals, qüestions socials i implicacions ambientals tecnològiques reconeix la importància crítica de la sostenibilitat en aquest camp. Si ens fixem en la tendència de les empreses tecnològiques líders, moltes són ben conscients dels problemes socials, econòmics i ambientals als quals s'enfronta la societat actual, i reconeixen que aquesta disciplina no pot existir aïllada d'aquests reptes globals. A més, s'han d'investigar els mètodes i les eines utilitzades per estimar la viabilitat econòmica del projecte per assegurar-se que són coherents amb els objectius de sostenibilitat. La gestió de recursos és un component crític del desenvolupament a llarg termini i mereix una consideració especial en el context dels projectes informàtics.

En aquesta època de major consciència sobre els problemes socials, econòmics i ambientals, no es pot exagerar el paper dels productes i serveis de la informàtica a l'hora d'exacerbar o mitigar aquests problemes. L'innegable impacte mediambiental del sector, així com les possibles conseqüències per a la salut, la seguretat i la justícia social derivades dels projectes i accions de la informàtica, posen de manifest la necessitat de posar més èmfasi en la sostenibilitat.

Finalment, quan s'aprofundeix en projectes, productes i serveis informàtics, és fonamental reconèixer la complexa xarxa d'interaccions que es produeixen amb altres actors en processos, activitats i projectes més grans, ja que aquestes interaccions tenen un impacte significatiu dels nostres esforços. En essència, aquesta reflexió estableix les bases per a la investigació sobre la sostenibilitat informàtica, destacant la seva naturalesa polifacètica i les implicacions per a la nostra societat.

Pretenem ampliar aquests punts en els apartats següents, discutint les dimensions econòmica, ambiental i social del projecte. Aquesta investigació pretén demostrar no només una comprensió a fons d'aquests aspectes crítics, sinó també un compromís per incorporar els principis de sostenibilitat al nucli del treball.
