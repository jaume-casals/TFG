\section{Gestió del risc} \label{sec:gestio_risc}
A l'apartat d'obstacles i riscos potencials \ref{ssec:abast-riscos} s'ha mencionat alguns dels obstacles potencials que poden sorgir. Gràcies a aquestes prediccions, ha sigut possible anticipar els possibles retards i dificultats que provoquin aquests punts. Naturalment, la gestió de riscs s'ha treballat com un procés continu i han sigut buscats en cada part de totes les fases durant el desenvolupament del projecte.

No només es mencionaran els riscos, sinó que es detallarà, per cadascun d'ells, quines són les mesures, plans secundaris, efectes en els recursos o bloquejos temporals que podrien provocar. En el cas del temps estimat extra, s'arrodoniran les hores al següent enter. A continuació es detallen els riscos principals:


\begin{itemize}
    \item \textbf{Models insuficients:} A causa de les limitacions per aquest projecte, es va preveure que no hi hauria un gran nombre de models del llenguatge natural que puguin satisfer tots els requisits que es necessiten. El fet que la informació amb la qual s'ha treballat sigui confidencial i l'alta potència que requereixen les solucions que s'ha plantejat, crea una barrera entre tots els models disponibles i els que es poden utilitzar. És important mencionar, que el problema amb el qual s'ha treballat és d'alta complexitat i, es va plantejar la possibilitat, que els resultats amb el millor model disponible fossin insuficients pels estàndards establerts. En última instància, també es va plantejar una inversió econòmica si fos necessari per accedir a un model privat. Aquest risc recau sobre les tasques TP2 i TP3 amb una extensió estimada del temps del 15\% (9 hores).
    \item \textbf{Diversitat de les dades:} La diversitat dels documents disponibles ha fet necessària una decisió estratègica per racionalitzar l'enfocament i refinar l'abast del projecte. En conseqüència, s'ha optat per usar exclusivament tiquets que tractin sobre amenaces de \textit{phishing}. A més a més, es va demanar una reducció del nombre de camps que s'extreuen dels tiquets amb el qual es va consensuar l'extracció només els set camps més importants. Aquest enfocament específic no només ha garantit una anàlisi més coherent, sinó que també ha ofert l'oportunitat d'extreure idees més significatives de les dades disponibles, millorant en última instància la qualitat i el rigor del treball. Aquest risc recau sobre les tasques D1.1 i D1.2 amb una extensió estimada del temps del 5\% (2 hores).
    \item \textbf{No tenir suficient informació:} Com s'ha observat en els primers passos del projecte, hi ha hagut una mancança de tiquets per poder extreure informació. Des del desconeixement, s'esperava que hi hagués un flux més estable de dades pròximament, però, en cas contrari, es destinaria més temps a recopilar els tiquets necessaris, a generar casos sintètics o aconseguir un \textit{dataset} similar per assegurar la qualitat d'aquests. Aquest risc recau sobre la tasca D1.1 amb una extensió estimada del temps del 10\% (3 hores).
    \item \textbf{Falta d'experiència:} La complexitat d'aquest projecte ha plantejat dubtes sobre la capacitat de l'equip per dur a terme tasques tan intrincades. El tractament d'informació confidencial i la necessitat de solucions molt potents requereixen un nivell de competència elevat. Per minimitzar aquest risc, s'han organitzat sessions de formació addicionals i activitats d'intercanvi de coneixements. Això requereix assignar temps addicional perquè els membres de l'equip es familiaritzin amb les eines i tecnologies necessàries per al projecte. A més a més, per mitigar el dèficit d'experiència s'ha decidit consultar un especialista en la matèria per encaminar correctament el projecte. Aquest risc és especialment pertinent per a les tasques TP1 i TP2, amb un augment de temps estimat aproximat del 10\% (5 hores).
    \item \textbf{Limitació dels recursos computacionals:} La limitació dels recursos de maquinari, especialment a les GPU, ha plantejat reptes importants en l'àmbit de l'aprenentatge autònom. L'elecció del model adequat per a un projecte té un impacte significatiu en la qualitat dels resultats i el temps d'entrenament. La manca de recursos limita les opcions, cosa que presenta reptes i riscos. En concret aquest risc afecta les tasques TP2 i D2.2 a causa d'haver d'explorar més models i a haver de dedicar més temps a la inferència i afinament de cadascun d'ells (inclosa l'elecció final) afegint un possible increment del temps del 15\%.
\end{itemize}
