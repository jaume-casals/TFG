\section{Recursos} \label{sec:recursos}

\subsection{Recursos humans}
Es defineixen els següents rols dins de l'equip:

\begin{itemize}
    \item \textbf{Cap del projecte:} Responsable de supervisar tot el cicle de vida del projecte. Respecta les guies d'estil i la coherència. Entre les seves funcions s'inclouen la planificació, l'organització de reunions d'equip i la garantia que el projecte avança segons els terminis establerts.
    \item \textbf{Desenvolupador júnior:} S'encarrega d'implementar la solució escollida. Les seves responsabilitats inclouen codificar i posar a prova el sistema per desenvolupar les característiques i les funcionalitats especificades. Hi participen dos desenvolupadors júniors en el projecte.
    \item \textbf{Expert científic:} Proporciona orientació experta sobre tècniques, algorismes i metodologies de processament del llenguatge natural. La seva funció és consultiva i contribueix assessorant sobre les millors pràctiques i aportant idees per millorar els aspectes computacionals del projecte. Es disposarà d'un únic 
\end{itemize}

\subsection{Recursos materials}
Aquests són els recursos materials que s'estima que seran necessaris per al correcte desenvolupament del projecte:

\begin{itemize}
    \item \textbf{Ordinador} amb els seus perifèrics.
    \item \textbf{Sala de reunions} per celebrar les reunions setmanals.
    \item \textbf{PyCharm} serà l'entorn de desenvolupament predilecte.
    \item \textbf{Git} per sistematitzar el control de les versions del codi.
    \item \textbf{Overleaf} per redactar la memòria.
    \item \textbf{onlinegantt.com} per l'elaboració del diagrama Gantt.
    \item \textbf{VPN} per accedir a les bases de dades i als servidors cedits per l'Agència.
    \item \textbf{Portàtil} cedit per l'Agència amb l'entorn configurat per ser el més confidencial possible.
    \item \textbf{Google Meet} per les reunions no presencials.
    \item \textbf{Google Drive} per l'emmagatzematge de documents relacionats amb el projecte.
\end{itemize}
