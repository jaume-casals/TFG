\chapter{Planificació temporal}

Segons el conveni de cooperació educativa acordat, aquest projecte s'ha iniciat el dia 16 de setembre de 2023 i ha conclòs el 19 de gener de 2024, amb una duració de setze setmanes i dos dies (en un espai temporal de quatre mesos i tres dies). Aquestes dates han sigut escollides per començar amb l'inici de GEP i finalitzar amb la lectura del Treball de Fi de Grau. Es treballen 5 hores al dia, tot i que, puntualment, es treballà fora de l'horari laboral. S'han extret els dies festius tals com les setmanes del 25 de desembre fins al 7 de gener per vacances de Nadal.

En total, es dediquen unes 600 hores en aquest projecte. Es dediquen 460 hores al treball previ i desenvolupament del projecte i 140 per la documentació i redacció d'aquesta memòria. Aquesta planificació temporal només és una estimació de les hores dedicades per l'autor, tot i que a l'equip hi participi més persones.

\import{./}{50-planificacio_temporal/51-descripcio_tasques}

\import{./}{50-planificacio_temporal/52-recursos}

\import{./}{50-planificacio_temporal/53-taula_tasques}

\import{./}{50-planificacio_temporal/54-gantt}

\import{./}{50-planificacio_temporal/55-gestio_risc}
