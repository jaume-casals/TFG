\pagenumbering{gobble} % Avoid page numbering

\selectlanguage{catalan}                        % Idioma índex
\tableofcontents
\listoftables
\listoffigures
\clearpage

\pagenumbering{arabic}

%% Index esperat:
\begin{comment}
            ANÀLISI DE TIQUETS D'INCIDÈNCIES MITJANÇANT PROCESSAMENT DEL LLENGUATGE NATURAL (NLP)
1 Contextualització i abast
    1.1 Contextualització 
        1.1.1 Context 
        1.1.2 Problema a resoldre 
        1.1.3 Actors implicats 
        [?] 1.1.4 Justificació
        1.1.5 Lleis i regulacions 
    1.2 Abast 
        1.2.1 Objectius 
        1.2.2 Requisits funcionals 
        1.2.3 Requisits no funcionals 
        1.2.4 Obstacles i riscos potencials 
    1.3 Metodologia i rigor 
        1.3.1 Metodologia 
        1.3.2 Eines 
2 Exploració teòrica
    2.1 Conceptes generals
        2.1.1 Anàlisi d'un tiquet
        2.1. etc.
    2.2 Aprenentatge autònom 
        2.2.1 Models en general
        2.2.2 Models NLP
    2.3 Estat de l'art
        2.3.1 Aplicacions comercials
        2.3.2 Propostes descartades
        2.3.3 Comprovació models disponibles
            Models destacats
        2.3.4 Justificació de la tria
3 Desenvolupament del sistema
    3.1 Arquitectura teòrica del sistema (pipeline)
    3.2 Creació dataset teòric
    [?] 3.3 Creació de tests
    3.4 Finetune teòric
    3.5 Dataset i finetune amb dades reals
        3.5.1 Estadístiques descriptives
        3.5.2 Tendències i patrons
        3.5.3 Problemes i incidències comunes
        3.5.4 Anomalies i valors atípics
    3.6 Desplegament API
4 Avaluació amb resultats reals
    4.1 Eficàcia de la solució i anàlisi de resultats
5 Planificació temporal
    5.1 Descripció de les tasques 
        5.1.1 Gestió de Projecte [GP] (180 hores) 
        5.1.2 Treball Previ [TP] (80 hores) 
        5.1.3 Desenvolupament [D] (340 hores) 
    5.2 Recursos 
        5.2.1 Recursos humans 
        5.2.2 Recursos materials 
    5.3 Taula de tasques 
    5.4 Diagrama de Gantt 
    5.5 Gestió del risc 
6 Gestió Econòmica
    6.1 Costos de personal i activitat 
    6.2 Costos genèrics 
        6.2.1 Amortitzacions 
        6.2.2 Consum elèctric 
        6.2.3 Connexió internet 
        6.2.4 Espai d'oficina 
        6.2.5 Total costos genèrics 
    6.3 Contingències 
    6.4 Imprevistos 
    6.5 Cost total del projecte 
    6.6 Control de gestió 
    [?] 6.7 Valoració econòmica final
7 Sostenibilitat
    7.1 Autoavaluació 
    7.2 Dimensió econòmica 
    7.3 Dimensió ambiental 
    7.4 Dimensió social 
8 Integració del coneixement
    8.1 Competències tècniques del projecte
    8.2 coneixement de les assignatures
9 Conclusions 
    9.1 Assoliment dels objectius
        9.1.1 Estudi de l'estat de l'art
        9.1.2 Sistema d'extracció d'informació
        9.1.3 Implementació del pipeline
        9.1.4 Desplegament API
    9.2 Treball futur
    9.3 Conclusions personals
Apèndixs
    A. Exemples de tiquets d'incidències
\end{comment}
