\section{Estat de l'art}
En aquesta secció es revisarà l'estat de l'art dels models que poden solucionar el problema plantejat. Per aconseguir la millor solució, s'ha proposat mantenir un rang de cerca ampli, i considerar moltes propostes que en un principi no havien sigut plantejades.
\subsection{Aplicacions comercials}
Lògicament, un model de generació del llenguatge natural pot ser una eina molt potent si es busca el propòsit adequat. És per això, que moltes empreses han decidit invertir en aquest sector per assolir aquests models tan potents. Aquests models són, generalment, l'estat de l'art en aquest àmbit, ja que disposen de molts recursos per obtenir el millor resultat possible. A continuació es presenten els models comercialitzats més importants en l'àmbit global:

per cada: nom, quin us té (chatbot, etc..) o en que es centra, empresa que ho fa, model de negoci (online, offline sobretot), etc?

\begin{itemize}
    \item \textbf{GPT-4:} \textit{GPT-4} és un gran model multimodal que pot agafar entrades d'imatge i text i produir sortides de text. Representa el darrer avenç en la investigació d'\textit{OpenAI} sobre l'ampliació del \textit{deep learning}. El seu principal ús és per crear xatbots (bots conversacionals) generals, que poden conversar sobre temes molt diversos i, fins i tot, cercar a la xarxa per aportar informació extra. Alguns exemples són \textit{ChatGPT} i \textit{Bing Chat}, tots dos són serveis en línia i, per tant, no aplicables per al projecte.
    \item \textbf{LaMDA:} \textit{LaMDA} és una col·lecció de grans models lingüístics desenvolupats per Google que treballen conjuntament per resoldre diverses tasques de generació de text. \textit{LaMDA} no està disponible públicament i només es pot accedir utilitzant la seva plataforma. Google va integrar aquest model en alguns dels seus productes, per millorar la seva funcionalitat i també va desenvolupar una primera versió d'un xatbot (\textit{Google Bard}) que durant un temps va usar aquest model.
    \item \textbf{PaLM 2:} \textit{PaLM 2} és un gran model lingüístic (LLM) que amplia el llegat de Google en matèria d'aprenentatge automàtic i de recerca responsable de la IA. S'ha entrenat en un corpus divers i multilingüe de text, codi, matemàtiques i pàgines web i va ser usat durant un temps com a xatbot amb \textit{Google Bard}. Tot i que no és de codi obert, es pot accedir a PaLM 2 a través d'una API.
    \item \textbf{Gemini:} \textit{Gemini} de Google és un ambiciós projecte d'intel·ligència artificial. És gran model de llenguatge (LLM) que tingui la capacitat de fer una gran varietat de tasques, incloses les relacionades amb text, imatges i àudio. \textit{Gemini} compta amb un disseny multimodal que permet la integració de diversos tipus d'informació i dona lloc a una àmplia gamma de formats de sortida. No és una plataforma de codi obert, per la qual cosa el codi i les dades no són accessibles al públic. Per altra banda, s'integra amb eines d'IA per a la generació de continguts, com ara la versió més nova de \textit{Google Bard}, que permet la interacció de l'usuari amb aquest model.
    \item \textbf{Claude 2:} \textit{Claude 2} és un model de llenguatge d'intel·ligència artificial avançat desenvolupat per \textit{Anthropic} amb un enfocament a IA segura i beneficiosa. Permet generar textos de primera qualitat i contextualment apropiats per a diverses aplicacions, com ara la generació de codi, l'anàlisi de textos i la redacció de composicions. És accessible al públic mitjançant un lloc web beta (en forma de xatbot), però el codi i les dades utilitzades per la formació i el funcionament no són accessibles pel públic.
    \item \textbf{Inflection-1:} \textit{Inflection-1} és un gran model de llenguatge (LLM) desenvolupat per \textit{Inflection AI}. És la base de \textit{Pi}, un assistent d'IA que pretén ser un xatbot de companyia, que continua les converses més llargues amb els usuaris i s'adapta a cadascun d'ells en funció d'aquestes converses. \textit{Inflection-1} no està disponible publicament i només és accessible en línia a través del xatbot que fa servir.
    \item \textbf{Grok-1:} \textit{Grok-1} va ser creat per \textit{xAI}. Es tracta del motor que impulsa Grok, un \textit{xatbot} capaç de respondre gairebé qualsevol dubte i fins i tot d'oferir suggeriments sobre les preguntes que s'han de formular. Grok està configurat per assistir als usuaris amb la xarxa social que del que és propietari el mateix fundador de l'empresa. Grok-1 no és un programa de codi obert i no es pot emprar de manera lliure a excepció d'una beta oberta.
\end{itemize}
\subsection{Propostes descartades}
\subsection{Comprovació models disponibles}
\subsection{Models destacats}
\subsection{Justificació de la tria}