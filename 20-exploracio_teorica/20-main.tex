\chapter{Exploració teòrica}
En aquesta secció es repassen totes aquelles idees necessàries per a una comprensió completa del treball. Es comença definint els conceptes fonamentals i explicant un tiquet d'incidències completament. Més endavant, es defineix l'aprenentatge autònom i com funciona, amb una atenció especial als models NLP. Per finalitzar, s'explora l'estat de l'art actual i com se solucionen els problemes de NLP avui en dia. En última instància, s'escull el model que serà usat per resoldre el problema plantejat.
\import{./}{20-exploracio_teorica/21-conceptes}

\import{./}{20-exploracio_teorica/22-apa}

\import{./}{20-exploracio_teorica/23-estat-de-lart}

\begin{comment}
2 Exploració teòrica
2.1 Conceptes generals
    2.1.1 Anàlisi d'un tiquet
    2.1. etc.
2.2 Aprenentatge autònom 
    2.2.1 Models en general
    2.2.2 Models NLP
2.3 Estat de l'art
    2.3.1 Aplicacions comercials
    2.3.2 Propostes descartades
    2.3.3 Comprovació models disponibles
    2.3.4 Models destacats
    2.3.5 Justificació de la tria
\end{comment}