\chapter{Exploració teòrica}

Aquest és un treball de final de grau del Grau d'Enginyeria Informàtica que s'imparteix a la Facultat d'Informàtica de Barcelona (FIB), que forma part de la Universitat Politècnica de Catalunya (UPC). El treball actual s'ha realitzat dintre d'un Conveni de Cooperació Educativa com a part d'un projecte dut a terme pel laboratori d'innovació i recerca inLab FIB, pertanyent a la Facultat d'Informàtica de Barcelona.

\import{./}{20-exploracio_teorica/21-conceptes}

\import{./}{20-exploracio_teorica/22-apa}

\import{./}{20-exploracio_teorica/23-estat-de-lart}

\begin{comment}
2 Exploració teòrica
2.1 Conceptes generals
    2.1.1 Anàlisi d'un tiquet
    2.1. etc.
2.2 Aprenentatge autònom 
    2.2.1 Models en general
    2.2.2 Models NLP
2.3 Estat de l'art
    2.3.1 Aplicacions comercials
    2.3.2 Propostes descartades
    2.3.3 Comprovació models disponibles
    2.3.4 Models destacats
    2.3.5 Justificació de la tria
\end{comment}