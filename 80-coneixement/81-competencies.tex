\section{Competències tècniques del projecte}
En aquesta secció es descriuen els aspectes associats a les competències tècniques descrites a la Facultat d'Informàtica de Barcelona dins del projecte i es fa una anàlisi del nivell d'assoliment de cadascuna.

\subsection*{CCO1.1}
\begin{quote}
    Avaluar la complexitat computacional d'un problema, conèixer estratègies algorísmiques que puguin dur a la seva resolució, i recomanar, desenvolupar i implementar la que garanteixi el millor rendiment d'acord amb els requisits establerts.
\end{quote}
S'ha estudiat el problema i el camp al qual pertany i s'ha après sobre les tècniques de \textit{deep learning} més recents per la resolució dels objectius plantejats. A més a més, s'ha realitzat un estudi de l'estat de l'art per tal d'aconseguir la solució amb el millor rendiment possible. Es considera que aquesta competència \textbf{ha sigut assolida satisfactòriament}.

\subsection*{CCO2.1}
\begin{quote}
    Demostrar coneixement dels fonaments, dels paradigmes i de les tècniques pròpies dels sistemes intel·ligents, i analitzar, dissenyar i construir sistemes, serveis i aplicacions informàtiques que utilitzin aquestes tècniques en qualsevol àmbit d'aplicació.
\end{quote}
S'ha comprès els models NLP i el coneixement de les tècniques de \textit{fine-tuning} que després s'ha aplicat per crear l'aplicació al sistema de tiquets d'incidències. Es considera que aquesta competència \textbf{ha sigut assolida satisfactòriament}.

\subsection*{CCO2.2}
\begin{quote}
    Capacitat per a adquirir, obtenir, formalitzar i representar el coneixement humà d'una forma computable per a la resolució de problemes mitjançant un sistema informàtic en qualsevol àmbit d'aplicació, particularment en els que estan relacionats amb aspectes de computació, percepció i actuació en ambients o entorns intel·ligents.
\end{quote}
Els models NLP han sigut entrenats a partir de textos generats per humans. El procés ha consistit a formalitzar i representar aquests coneixements de manera computable per resoldre problemes a l'àmbit de la ciberseguretat. Es considera que aquesta competència \textbf{ha sigut assolida satisfactòriament}.

\subsection*{CCO2.4}
\begin{quote}
    Demostrar coneixement i desenvolupar tècniques d'aprenentatge computacional; dissenyar i implementar aplicacions i sistemes que les utilitzin, incloent les que es dediquen a l'extracció automàtica d'informació i coneixement a partir de grans volums de dades.
\end{quote}
La tèncica d'aprenentatge computacional ha sigut, principalment el \textit{deep learning}. S'ha demostrat el coneixement sobre la tècnica i s'ha implementat un sistema amb la funció principal d'extreure informació automàticament. Aquest sistema ha sigut desenvolupat amb un gran volum de dades d'entrenament per l'aprenentatge del model. Es considera que aquesta competència \textbf{ha sigut assolida satisfactòriament}.

\subsection*{CCO3.1}
\begin{quote}
    Implementar codi crític seguint criteris de temps d'execució, eficiència i seguretat.
\end{quote}
En treballar amb tiquets d'incidents per a una agència de ciberseguretat, és crucial donar prioritat a la seguretat de tot el sistema. Això implica l'aplicació de pràctiques el maneig adequat de la informació delicada i la protecció davant de possibles vulnerabilitats. Es considera que aquesta competència \textbf{ha sigut assolida satisfactòriament}.

\subsection*{CCO3.2}
\begin{quote}
    Programar considerant l'arquitectura hardware, tant en asemblador com en alt nivell.
\end{quote}
En aquest projecte, l'arquitectura del sistema ha determinat no només les possibles tècniques a desenvolupar, sinó que també el rendiment final del sistema. Addicionalment, la dependència de les tècniques de \textit{deep leaning} a les GPU, ha obligat a la consideració de l'arquitectura amb la qual es treballa. Es considera que aquesta competència \textbf{ha sigut assolida satisfactòriament}.
