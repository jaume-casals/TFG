\section{Coneixement de les assignatures}
\subsection*{Aprenentatge Automàtic}
En aquesta assignatura s'estudien diverses tècniques de modelatge, entre els quals s'inclouen les xarxes neuronals artificials i altres sistemes d'aprenentatge autònom. Addicionalment, també ha permès familiaritzar-se amb els ambients de \textit{Jupyter Notebook}.

\subsection*{Intel·ligència Artifical}
L'assignatura ofereix una àmplia panoràmica del camp, el caràcter multidisciplinari, el desenvolupament històric i les aplicacions actuals. Ha permès un enteniment general de la intel·ligència artificial i de les seves aplicacions no només a la teoria, sinó que també aplicades de forma pràctica.

\subsection*{Llenguatges de programació}
El contingut de l'assignatura no contribueix directament a la tasca de l'anàlisi automàtica de tiquets d'incidents, però atès que \textit{Python} és el llenguatge de programació principal per al desenvolupament del sistema, la informació específica sobre aquest llenguatge es pot aplicar directament al projecte. Els coneixements sobre la sintaxi, les biblioteques i les millors pràctiques de Python han ajudat a implementar el model \textit{NLP}.

\subsection*{Probabilitat i Estadística}
El contingut d'aquesta assignatura ha resultat beneficiosa per al projecte. Comprendre com modelar fenòmens aleatoris ha sigut important a l'hora de tractar els textos d'entrada. Una comprensió dels models de probabilitat ha ajudat a analitzar correctament les dades i els resultats dels models.
