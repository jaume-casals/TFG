\section{Anàlisi dels resultats amb el \textit{dataset} real}

A la secció anterior, s'ha explorat el rendiment de diversos models del llenguatge natural en el \textit{dataset} sintètic. Aquest nou conjunt de dades presenta un salt significatiu en complexitat en comparació del primer. Aquest \textit{dataset} supera els límits del que els models anteriors podien manejar. Malauradament, quan es van entrenar amb aquestes noves dades, els models anteriors van obtenir resultats decebedors.

Per tant, en aquesta secció s'ha seleccionat un nou conjunt de models escollits per fer front als reptes que planteja aquest \textit{dataset} més complex. Aquests models tenen una potència més gran de processament i capacitats avançades, cosa que els fa molt més adequats per extreure els camps dels tiquets.

\subsection{FLOR 6.3B (LoRA)}
Es va utilitzar la tècnica de \textit{LoRA} per l'entrenament, ja que permet que aquest gran model càpiga dins de la memòria de la GPU. L'entrenament va durar al voltant de 10 hores pel fet que la versió GPTQ (més optimitzada per GPU) no estava disponible. Addicionalment, cal mencionar que les dades de sortida tenien un format tipus JSON. Això no obstant, a l'hora d'avaluar els resultats del model entrenat, el model només continuava el tiquet de l'input amb dades addicionals que no existeixen en el tiquet original. Per tant, aquest model no va treure els resultats i la qualitat que s'esperava. En conseqüència, es va decidir descartar aquest model, ja que només són vàlids els resultats que surtin de forma organitzada com un JSON.

Per a comprovar el rendiment del model és comú crear o separar una secció del \textit{dataset} que el model no hagi vist mai durant l'entrenament. Aquestes dades són un exemple de les dades reals que es podrien trobar quan aquest sistema es posi en funcionament. Aquest \textit{dataset} s'anomena "test dataset". El \textit{dataset} que s'ha creat conté 15 tiquets que són representatius del \textit{dataset} original. Després s'ha fet inferència sobre aquests tiquets i s'ha comparat el resultat amb el resultat esperat. Per a comprovar el resultat amb el resultat esperat s'ha fet servir l'índex de Jaccard, també conegut com a similitud de Jaccard. Aquesta similitud és un indicatiu del grau d'encert del model.

Es va entrenar el model amb les dades del \textit{dataset} amb errors \ref{tab:resultat_entrenament_flor_1} i el que no en contenia \ref{tab:resultat_entrenament_flor_2} i ambdós mostren que no és capaç d'entendre la tasca i, per tant, té un encert del 0\% en qualsevol comprovació es faci.

{\setlength{\tabcolsep}{1pt}   %Espai a la dreta i esquerra de cada cel·la amb text
\begin{table}[H]
    \centering
    \begin{tabular}{lcccccccc}
        \toprule
        Mètrica 
        & \begin{tabular}{@{}c@{}}affected \\ Users\end{tabular}
        & \begin{tabular}{@{}c@{}}mitigation \\ Actions\end{tabular}
        & \begin{tabular}{@{}c@{}}control \\ Actions\end{tabular}
        & \begin{tabular}{@{}c@{}}urlMail \\ Incident\end{tabular}
        & \begin{tabular}{@{}c@{}}fromMail \\ Incident\end{tabular}
        & \begin{tabular}{@{}c@{}}recipientMail \\ Incident\end{tabular}
        & \begin{tabular}{@{}c@{}}subjectMail \\ Incident\end{tabular} 
        & Total \\         
        \midrule
        \makecell[l]{ Accuracy \\ \textit{Jaccard} \\ \textit{similarity} } & 0.0\% & 0.0\% & 0.0\% & 0.0\% & 0.0\% & 0.0\% & 0.0\% & 0.0\% \\ 
        \makecell[l]{ NotFound \\ \textit{pred} } & 0.0\% & 0.0\% & 0.0\% & 0.0\% & 0.0\% & 0.0\% & 0.0\% & 0.0\% \\ 
        \bottomrule
    \end{tabular}
    \caption[Avaluació del model \textit{FLOR 6.3B}]{Estadístiques dels resultats de l'entrenament del model \textit{FLOR 6.3B} amb \textit{LoRA} amb el primer conjunt de dades }
    \label{tab:resultat_entrenament_flor_1}
\end{table}
}

{\setlength{\tabcolsep}{1pt}   %Espai a la dreta i esquerra de cada cel·la amb text
\begin{table}[H]
    \centering
    \begin{tabular}{lcccccccc}
        \toprule
        Mètrica 
        & \begin{tabular}{@{}c@{}}affected \\ Users\end{tabular}
        & \begin{tabular}{@{}c@{}}mitigation \\ Actions\end{tabular}
        & \begin{tabular}{@{}c@{}}control \\ Actions\end{tabular}
        & \begin{tabular}{@{}c@{}}urlMail \\ Incident\end{tabular}
        & \begin{tabular}{@{}c@{}}fromMail \\ Incident\end{tabular}
        & \begin{tabular}{@{}c@{}}recipientMail \\ Incident\end{tabular}
        & \begin{tabular}{@{}c@{}}subjectMail \\ Incident\end{tabular} 
        & Total \\         
        \midrule
        \makecell[l]{ Accuracy \\ \textit{Jaccard} \\ \textit{similarity} } & 0.0\% & 0.0\% & 0.0\% & 0.0\% & 0.0\% & 0.0\% & 0.0\% & 0.0\% \\ 
        \makecell[l]{ NotFound \\ \textit{pred} } & 0.0\% & 0.0\% & 0.0\% & 0.0\% & 0.0\% & 0.0\% & 0.0\% & 0.0\% \\ 
        \bottomrule
    \end{tabular}
    \caption[Avaluació del model \textit{FLOR 6.3B} amb el segon conjunt de dades]{Estadístiques dels resultats de l'entrenament del model \textit{FLOR 6.3B} amb \textit{LoRA} amb el segon conjunt de dades }
    \label{tab:resultat_entrenament_flor_2}
\end{table}
}


\subsection{QWEN1.5-Instructed 7B (QLoRA)}

Els resultats que es poden observar pel model \textit{QWEN1.5} són molt interessants, perquè en el primer entrenament \ref{tab:resultat_entrenament_qwen_1} es pot veure que obté un encert general del 64,2\% i un 66,6\% de les seves respostes són "NotFound". En canvi, en el segon entrenament amb el \textit{dataset} corregit i que conté més "NotFound", el \textit{QWEN1.5} aconsegueix un encert del 71,4\% amb un 89,5\% de les respostes buides.

Això significa que la gran majoria de les vegades, el model contesta la mateixa resposta i és encertat com a correcte. Aquest resultat, comparat només numèricament és un bon resultat, però s'ha de tenir en compte què s'assoleix a costa de sacrificar més variabilitat del model i deixar a part altres casos.


{\setlength{\tabcolsep}{1pt}   %Espai a la dreta i esquerra de cada cel·la amb text
\begin{table}[H]
    \centering
    \begin{tabular}{lcccccccc}
        \toprule
        Mètrica 
        & \begin{tabular}{@{}c@{}}affected \\ Users\end{tabular}
        & \begin{tabular}{@{}c@{}}mitigation \\ Actions\end{tabular}
        & \begin{tabular}{@{}c@{}}control \\ Actions\end{tabular}
        & \begin{tabular}{@{}c@{}}urlMail \\ Incident\end{tabular}
        & \begin{tabular}{@{}c@{}}fromMail \\ Incident\end{tabular}
        & \begin{tabular}{@{}c@{}}recipientMail \\ Incident\end{tabular}
        & \begin{tabular}{@{}c@{}}subjectMail \\ Incident\end{tabular} 
        & Total \\         
        \midrule
        \makecell[l]{ Accuracy \\ \textit{Jaccard} \\ \textit{similarity} } & 26.6\% & 63.3\% & 80.0\% & 80.0\% & 46.6\% & 66.6\% & 86.6\% & 64.2\% \\ 
        \makecell[l]{ NotFound \\ \textit{pred} } & 20.0\% & 60.0\% & 86.6\% & 93.3\% & 33.3\% & 86.6\% & 86.6\% & 66.6\% \\ 
        \bottomrule
    \end{tabular}
    \caption[Avaluació del model \textit{QWEN1.5 7B}]{Estadístiques dels resultats de l'entrenament del model \textit{QWEN1.5 7B} amb QLoRA amb el primer conjunt de dades}
    \label{tab:resultat_entrenament_qwen_1}
\end{table}
}

{\setlength{\tabcolsep}{1pt}   %Espai a la dreta i esquerra de cada cel·la amb text
\begin{table}[H]
    \centering
    \begin{tabular}{lcccccccc}
        \toprule
        Mètrica 
        & \begin{tabular}{@{}c@{}}affected \\ Users\end{tabular}
        & \begin{tabular}{@{}c@{}}mitigation \\ Actions\end{tabular}
        & \begin{tabular}{@{}c@{}}control \\ Actions\end{tabular}
        & \begin{tabular}{@{}c@{}}urlMail \\ Incident\end{tabular}
        & \begin{tabular}{@{}c@{}}fromMail \\ Incident\end{tabular}
        & \begin{tabular}{@{}c@{}}recipientMail \\ Incident\end{tabular}
        & \begin{tabular}{@{}c@{}}subjectMail \\ Incident\end{tabular} 
        & Total \\         
        \midrule
        \makecell[l]{ Accuracy \\ \textit{Jaccard} \\ \textit{similarity} } & 53.3\% & 80.0\% & 93.3\% & 86.6\% & 53.3\% & 60.0\% & 73.3\% & 71.4\% \\ 
        \makecell[l]{ NotFound \\ \textit{pred} } & 80.0\% & 93.3\% & 100.0\% & 100.0\% & 80.0\% & 86.6\% & 86.6\% & 89.5\% \\ 
        \bottomrule
    \end{tabular}
    \caption[Avaluació del model \textit{QWEN1.5 7B} amb el segon conjunt de dades]{Estadístiques dels resultats de l'entrenament del model \textit{QWEN1.5 7B} amb QLoRA amb el conjut de dades corregit}
    \label{tab:resultat_entrenament_qwen_2}
\end{table}
}
