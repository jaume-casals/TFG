\section{Anàlisi dels resultats}
En aquesta secció s'ofereix una anàlisi dels resultats obtinguts amb els models que han sigut entrenats anteriorment amb el procediment de \textit{fine-tuning} explicat a la secció \ref{sec:fine-tune}.

Alguns resultats s'han perdut durant el procés d'entrenament. Aquesta pèrdua de dades és deguda a les característiques de la plataforma utilitzada per executar el codi, els quaderns Jupyter, que tant en un entorn local o mitjançant \textit{Google Colab} és possible no assolir aquests resultats si no es prenen les mesures adequades. Malauradament, aquests resultats d'entrenament perduts no es poden recuperar sense tornar a entrenar els models. Aquest desafiament inesperat ha demostrat la importància de la transparència en el procés de recerca, tot reconeixent tant els èxits com els reptes trobats durant l'experimentació.

En cada secció es poden veure dos tipus de taules:
\begin{itemize}
    \item \textbf{Taula d'entrenament:} Es mostra el progrés que s'ha aconseguit guardar de cada entrenament. Al final de cada \textit{Epoch} s'indiquen les mètriques de \textit{Train\_Loss} (escurçat a \textit{Train}), \textit{Validation\_Loss} (escurçat a Val), \textit{ROUGE-1}, \textit{ROUGE-2}, \textit{ROUGE-L} i \textit{ROUGE-Lsum}.
    \item \textbf{Taula d'avaluació:} Es fa una recopilació dels resultats obtinguts amb cada \textit{checkpoint} o punt de control que s'ha recuperat. A part de les mètriques de \textit{ROUGE}, es va afegir la mètrica de ``Coincidència exacta'' (escurçat a ``Exacte'') que simplement indica el percentatge de resultats generats que han coincidit exactament amb el resultat real.
\end{itemize}

\subsection{Flan-T5-Base}
L'entrenament inicial de la versió base del model \textit{Flan-T5} es resumeix a la taula següent, que mostra les mètriques clau durant les primeres \textit{Epochs}. 

\begin{table}[H]
    \centering
    \begin{tabular}{lcccccc}
    \toprule
    Epoch & Train & Val & ROUGE-1 & ROUGE-2 & ROUGE-L & ROUGE-Lsum \\
    \midrule
    1 & 0,058 & 0,045 & 94,662 & 92,672 & 93,843 & 93,955 \\
    2 & 0,037 & 0,036 & 96,162 & 94,540 & 95,625 & 95,725 \\
    3 & ... & ... & ... & ... & ... & ... \\
    \bottomrule
    \end{tabular}
    \caption[Primer entrenament del model \textit{Flan-T5-Base}]{Primer entrenament del model \textit{Flan-T5-Base} (amb etiquetes)}
\end{table}

Després d'aquest entrenament inicial, el model es va sotmetre a dos cicles addicionals. Aquest segon entrenament s'ha realitzat amb la segona funció de postprocessament, la qual no té en compte les etiquetes a l'hora de calcular l'error amb les diferents mètriques.

Els resultats de l'avaluació després d'aquest període d'entrenament estès es detallen a la taula següent.

\begin{table}[H]
    \centering
    \begin{tabular}{lcccccc}
    \toprule
    Model & ROUGE-1 & ROUGE-2 & ROUGE-L & ROUGE-Lsum & Exacte \\
    \midrule
    Model base   & 0,00  & 0,00  & 0,00  & 0,00  & 0,00\% \\
    Checkpoint 2 & 95,84 & 94,27 & 95,19 & 95,20 & 61,82\% \\
    Checkpoint 3 & 96,70 & 95,21 & 96,20 & 96,21 & 64,99\% \\
    Checkpoint 4 & 95,97 & 94,50 & 95,37 & 95,38 & 64,24\% \\
    Checkpoint 5 & 96,57 & 95,17 & 96,05 & 96,09 & 66,01\% \\
    \bottomrule
    \end{tabular}
    \caption[Avaluació del model \textit{Flan-T5-Base}]{Avaluació del model \textit{Flan-T5-Base} (sense etiquetes)}
\end{table}

Observant el resultat, es va poder veure que el model va mostrar una millora del rendiment necessària, com indica l'augment de les puntuacions \textit{ROUGE} als punts de control següents. La puntuació de coincidència exacta, una mesura de la capacitat del model per proporcionar respostes literals, també mostra una tendència positiva, assolint el seu punt màxim al \textit{Checkpoint 5}. La raó per la qual es creu que el model va disminuir el seu rendiment en finalitzar el \textit{Checkpoint 4}, és perquè va ser la primera \textit{epoch} on el model s'avaluava sense les etiquetes i això pot haver causat una disminució de la puntuació, però no necessàriament de la qualitat del model.





\subsection{Flan-T5-Base (LaMini)}
L'entrenament inicial, com es mostra a la taula següent, dona resultats prometedors després de només la segona \textit{epoch}.
\begin{table}[H]
    \centering
    \begin{tabular}{lcccccc}
    \toprule
    Epoch & Train & Val & ROUGE-1 & ROUGE-2 & ROUGE-L & ROUGE-Lsum \\
    \midrule
    1 & ...   & ...   & ...    & ...    & ...    & ... \\
    2 & 0,013 & 0,022 & 97,800 & 96,553 & 97,248 & 97,350 \\
    \bottomrule
    \end{tabular}
    \caption[Primer entrenament del model \textit{Flan-T5-Base (LaMini)}]{Primer entrenament del model \textit{Flan-T5-Base (LaMini)} (amb etiquetes)}
\end{table}

L'entrenament posterior sense etiquetes mostra un refinament més gran de les capacitats del model, amb la quarta i cinquena \textit{epochs} mostrant les següents tendències. Es pot observar un lleuger empitjorament de les mètriques, que segurament és degut al canvi a no utilitzar les etiquetes pel càlcul d'aquestes.

\begin{table}[H]
    \centering
    \begin{tabular}{lcccccc}
    \toprule
    Epoch & Train & Val & ROUGE-1 & ROUGE-2 & ROUGE-L & ROUGE-Lsum \\
    \midrule
    3 & ...  & ...    & ...    & ...    & ...    & ... \\
    4 & 0,012 & 0,029 & 96,594 & 95,267 & 96,109 & 96,136 \\
    5 & 0,005 & 0,029 & 96,794 & 95,538 & 96,257 & 96,290 \\
    \bottomrule
    \end{tabular}
    \caption[Segon entrenament del model \textit{Flan-T5-Base (LaMini)}]{Segon entrenament del model \textit{Flan-T5-Base (LaMini)} (sense etiquetes)}
\end{table}

La taula següent resumeix els resultats de l'avaluació, que demostren la millora contínua del model a través dels diversos \textit{checkpoints}.

\begin{table}[H]
    \centering
    \begin{tabular}{lcccccc}
    \toprule
    Model & ROUGE-1 & ROUGE-2 & ROUGE-L & ROUGE-Lsum & Exacte \\
    \midrule
    Model base   & 0,00  & 0,00  & 0,00  & 0,00  & 0,00\% \\
    Checkpoint 1 & 95,31 & 93,48 & 94,76 & 94,75 & 58,28\% \\
    Checkpoint 2 & 96,74 & 95,37 & 96,18 & 96,18 & 67,87\% \\
    Checkpoint 3 & 96,39 & 95,03 & 95,90 & 95,90 & 66,66\% \\
    Checkpoint 4 & 96,55 & 95,22 & 96,08 & 96,06 & 66,75\% \\
    Checkpoint 5 & 96,75 & 95,45 & 96,19 & 96,19 & 69,36\% \\
    \bottomrule
    \end{tabular}
    \caption[Avaluació del model \textit{Flan-T5-Base (LaMini)}]{Avaluació del model \textit{Flan-T5-Base (LaMini)} (sense etiquetes)}
\end{table}

Els resultats mostren que tot i que algunes mètriques no han progressat com s'esperava, hi ha hagut un increment en la valoració de coincidències exactes que asseguren que, finalment, el \textit{fine-tuning} ha sigut exitós.



\subsection{Flan-T5-Small}
A la taula següent es resumeix el procés d'entrenament del model \textit{Flan-T5-Small} al llarg de cinc \textit{epochs}.
\begin{table}[H]
    \centering
    \begin{tabular}{lcccccc}
    \toprule
    Epoch & Train & Val & ROUGE-1 & ROUGE-2 & ROUGE-L & ROUGE-Lsum \\
    \midrule
    1 & 0,084 & 0,041 & 94,86 & 92,50 & 93,70 & 93,84 \\
    2 & 0,055 & 0,036 & 96,13 & 93,95 & 94,94 & 95,06 \\
    3 & 0,043 & 0,033 & 96,61 & 94,74 & 95,59 & 95,73 \\
    4 & 0,040 & 0,032 & 96,62 & 94,81 & 95,74 & 95,89 \\
    5 & 0,037 & 0,032 & 96,58 & 94,78 & 95,71 & 95,86 \\
    \bottomrule
    \end{tabular}
    \caption[Entrenament del model \textit{Flan-T5-Small}]{Entrenament del model \textit{Flan-T5-Small} (amb etiquetes)}
\end{table}
El progrés del model durant l'entrenament indica una disminució dels errors tant a \textit{Train} com a \textit{Val} al llarg de les cinc \textit{epochs}, cosa que suggereix que ha après eficaçment de les dades etiquetades. Per altre banda, hi ha hagut una lleugera disminució de les mètriques de \textit{ROUGE} durant l'última \textit{epoch}, que podrien donar a entendre que el penúltim \textit{checkpoint} pot tenir millor rendiment que l'últim. Aquesta última afirmació va ser contrastada comprovant l'avaluació dels punts de control mostrats a la següent taula. 

\begin{table}[H]
    \centering
    \begin{tabular}{lcccccc}
    \toprule
    Model & ROUGE-1 & ROUGE-2 & ROUGE-L & ROUGE-Lsum & Exacte \\
    \midrule
    Model base   & 0,00  & 0,00  & 0,00  & 0,00  & 0,00\% \\
    Checkpoint 4 & 95,07 & 93,20 & 94,40 & 94,40 & 54,09\% \\
    Checkpoint 5 & 95,07 & 93,17 & 94,39 & 94,39 & 54,74\% \\
    \bottomrule
    \end{tabular}
    \caption[Avaluació del model \textit{Flan-T5-Small}]{Avaluació del model \textit{Flan-T5-Small} (sense etiquetes)}
\end{table}

Després	de l'avaluació sense etiquetes, s'ha pogut comprovar que el model ha decidit comprometre una lleguera part de puntuació \textit{ROUGE} per aconseguir una última millora de les coincidències exactes.



\subsection{Flan-T5-Small (LaMini)}
La taula següent presenta el progrés de l'entrenament del model Flan-T5-Small amb el \textit{fine-tuning} \textit{LaMini}, incloses les mètriques clau de l'error de \textit{Train} i \textit{Val}, així com les puntuacions \textit{ROUGE}.
\begin{table}[H]
    \centering
    \begin{tabular}{lcccccc}
    \toprule
    Epoch & Train & Val & ROUGE-1 & ROUGE-2 & ROUGE-L & ROUGE-Lsum \\
    \midrule
    1 & ... & ... & ... & ... & ... & ... \\
    2 & ... & ... & ... & ... & ... & ... \\
    3 & 0,066 & 0,043 & 95,55 & 92,75 & 93,87 & 94,12 \\
    4 & 0,064 & 0,042 & 95,61 & 92,88 & 93,93 & 94,19 \\
    5 & ... & ... & ... & ... & ... & ... \\
    \bottomrule
    \end{tabular}
    \caption[Entrenament del model \textit{Flan-T5-Small (LaMini)}]{Entrenament del model \textit{Flan-T5-Small (LaMini)} (amb etiquetes)}
\end{table}

Després de l'entrenament continuat, es va avaluar el model sense etiquetes. Els resultats es resumeixen a la taula següent.

\begin{table}[H]
    \centering
    \caption{Evaluació (sense etiquetes)}
    \begin{tabular}{lcccccc}
    \toprule
    Model & ROUGE-1 & ROUGE-2 & ROUGE-L & ROUGE-Lsum & Exacte \\
    \midrule
    Model base   & 0,00  & 0,00  & 0,00  & 0,00  & 0,00\% \\
    Checkpoint 1 & 90,27 & 86,79 & 88,56 & 88,59 & 31,09\% \\
    Checkpoint 2 & 92,51 & 89,48 & 91,03 & 91,05 & 37,80\% \\
    Checkpoint 3 & 92,83 & 89,96 & 91,39 & 91,41 & 40,31\% \\
    Checkpoint 4 & 93,36 & 90,62 & 92,07 & 92,08 & 43,01\% \\
    Checkpoint 5 & 93,48 & 90,77 & 92,16 & 92,16 & 43,38\% \\
    \bottomrule
    \end{tabular}
    \caption[Avaluació del model \textit{Flan-T5-Small (LaMini)}]{Avaluació del model \textit{Flan-T5-Small (LaMini)} (sense etiquetes)}
\end{table}

El model mostra una millora progressiva al llarg dels \textit{checkpoints} d'entrenament, amb puntuacions ROUGE i percentatges de ``coincidència exacta'' creixents.

\subsection{Flan-T5-Base (LoRA)}
El model Flan-T5-Base es va ajustar mitjançant el mètode LoRA (\textit{Low-rank Adaptation of LLMs}), que es va centrar a entrenar únicament les matrius ``q'', ``v'', ``k'' i ``o'' del mòdul d'atenció del model. Aquests paràmentres entrenables només representen l'1,41\% de tots els paràmetres del model, reduïnt en gran mesura la càrrega computacional.
\begin{table}[H]
    \centering
    \label{tab:additional-training}
    \begin{tabular}{lcccccc}
    \toprule
    Epoch & Train & Val & ROUGE-1 & ROUGE-2 & ROUGE-L & ROUGE-Lsum \\
    \midrule
    1 & 0,114 & 0,042 & 20,847 & 16,240 & 20,715 & 20,733 \\
    2 & 0,057 & 0,036 & 20,757 & 16,301 & 20,672 & 20,673 \\
    3 & 0,048 & 0,034 & 20,748 & 16,288 & 20,655 & 20,660 \\
    \bottomrule
    \end{tabular}
    \caption[Entrenament del model \textit{Flan-T5-Base} amb LoRA]{Entrenament del model \textit{Flan-T5-Base} amb LoRA (amb etiquetes)}
\end{table}
El \textit{fine-tuning} amb el mètode \textit{LoRA} mostra un patró clar en els errors de \textit{Train} i \textit{Val}, cosa que indica una millora gradual del rendiment del model al llarg de les \textit{epochs} especificades. No obstant això, les puntuacions ROUGE mostren una tendència diferent, cosa que suggereix que la capacitat del model per captar el matisos del \textit{dataset} i generar resultats rellevants es pot veure qüestionada en aquest entrenament. És per aquest motiu que es va decidir cancelar el entrenament i no continuar amb el procés.
