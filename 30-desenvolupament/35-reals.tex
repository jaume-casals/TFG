\section{Dataset i fine-tune amb dades reals}
En aquest projecte no es va obtenir accés a les dades reals fins al mes de gener de 2024. Aquesta data és un any després de l'inici del projecte. Durant aquest any va canviar tant els requisits d'\textbf{i2CAT}, els mètodes de desenvolupament i del sistema plantejats, els coneixements adquirits, els objectius del projecte i, fins i tot, la mentalitat dels desenvolupadors. És aquesta diferència temporal que fa que es vegi com un projecte diferent o l'inici del projecte amb l'arribada de les dades. A més a més, el temps, el maquinari i l'accés el software van ser tant limitants que es va haver de centrar-se en trobar una sol·lució per al projecte el més ràpid possible.

\subsection{Exploració de les dades}
Les dades reals consisteixen en una llista de 804 tiquets (700 originalment pactats + 104 de reserva) que van ser seleccionats manualment per \textbf{i2CAT}. D'aquests hi ha aproximadament la meitat d'incidents de \textit{Phishing} o \textit{Spam} i l'altre meitat de \textit{Malware}.

\begin{table}[H]
    \centering
    \begin{tabular}{lccr}
        \Xhline{2\arrayrulewidth}
        \textbf{Cua} & \textbf{Recompte} \\
        \hline
        Gestió Incidents (Malware) & 363 \\
        Protecció Perimetre (Phishing / Spam) & 441  \\
        \hline
        \textbf{Total} & \textbf{804} \\
        \Xhline{2\arrayrulewidth}
    \end{tabular}
    \caption{Recompte del nombre de tiquets per cua de la primera selecció de tiquets.}
    \label{tab:recompte_per_cua}
\end{table}

Després d'etiquetar 25 tiquets es va veure que ja majoria de les respostes eren \pyth{"NotFound"}. Això és una mala notícia perquè llavors el model aprendrà que per obtenir molt bon rendiment haurà de contestar sempre el mateix. Mirant més detingudament, es va descobrir que la majoria dels tiquets on suceeix aquest problema són els tiquets de \textit{Malware} ja que les entitats de sortida no estan fetes per tenir en comptes aquest tipus de tiquets.

Com a resposta, es va plantejar quina seria la millor sol·lució per arreglar aquest problema. Es va plantejar afegir nous camps o crear un model diferent per cada cua. Degut a la manca de temps per poder executar cap de les dues sol·lucions, \textbf{i2CAT} va seleccionar 461 tiquets dels 804 previs. D'aquesta nova selecció la majoria són de la cua de Protecció Perimetre tot i que hi ha alguns de Gestió Incidents.


\begin{table}[H]
  \centering
  \begin{tabular}{lccr}
      \Xhline{2\arrayrulewidth}
      \textbf{Cua} & \textbf{Recompte} \\
      \hline
      Gestió Incidents (Malware) & 59 \\
      Protecció Perimetre (Phishing / Spam) & 402  \\
      \hline
      \textbf{Total} & \textbf{461} \\
      \Xhline{2\arrayrulewidth}
  \end{tabular}
  \caption{Recompte del nombre de tiquets per cua de la segona selecció de tiquets.}
  \label{tab:recompte_per_cua_2}
\end{table}

% Boxplot + histograma de tiquets i tiquets + adjunts + referències
% Histograma només tiquets dividit per cua
% Histograma nombre articles d'adjunts i nombre referències