\section{Desplegament API}

Per l'API només es necessitava un sistema senzill que posés en marxa el sistema sense haver d'estar present. És per això que només s'ha implementat dos senzills \textit{endpoints}:

\begin{itemize}
    \item \texttt{processar\_excel}: Rep un fitxer excel i una columna i processa tots els nombres de tiquets de la columna a través de la \textit{pipeline}.
    \item \texttt{processar\_tiquet}: Rep un nombre de tiquet i el processa a través de la \textit{pipeline}.
\end{itemize}

S'ha definit una carpeta temporal on s'emmagatzemaran els fitxers pujats. A l'\textit{endpoint} \/processar\_excel\/, es defineix un gestor de peticions POST que espera un fitxer Excel(UploadFile) i un nombre de columna(int). Es desa el fitxer carregat en l'ubicació temporal i, a continuació, s'afegeix una tasca en segon pla per executar la funció d'execució de l'excel amb el nom de fitxer i el nombre de columna com a paràmetres. No seria necessari posar la funció en segon pla ja que encara que es facin més peticions a l'API amb el maquinari actual no es podria processar però així hi ha marge de millora en el futur.

A l'\textit{endpoint} \/processar\_tiquet\/, es defineix un altre gestor de peticions POST que espera un nombre(int). Simplement es crida a la funció que processa els tiquets individualment amb el nombre proporcionat i es retorna el resultat.
