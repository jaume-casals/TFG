\chapter{Contextualització i abast}

Aquest és un treball de final de grau del Grau d'Enginyeria Informàtica que s'imparteix a la Facultat d'Informàtica de Barcelona (FIB), que forma part de la Universitat Politècnica de Catalunya (UPC). El treball actual s'ha realitzat dintre d'un Conveni de Cooperació Educativa com a part d'un projecte dut a terme pel laboratori d'innovació i recerca inLab FIB, pertanyent a la Facultat d'Informàtica de Barcelona.

\import{./}{10-contextualitzacio_i_abast/11-contextualitzacio}

\import{./}{10-contextualitzacio_i_abast/12-abast}

\import{./}{10-contextualitzacio_i_abast/13-metodologia_i_rigor}

\begin{comment}
1 Contextualització i abast
1.1 Contextualització 
    1.1.1 Context 
    1.1.2 Problema a resoldre 
    1.1.3 Actors implicats 
    [?] 1.1.4 Justificació
    1.1.5 Lleis i regulacions 
1.2 Abast 
    1.2.1 Objectius 
    1.2.2 Requeriments funcionals 
    1.2.3 Requeriments no funcionals 
    1.2.4 Obstacles i riscos potencials 
1.3 Metodologia i rigor 
    1.3.1 Metodologia 
    1.3.2 Eines 
\end{comment}