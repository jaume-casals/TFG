\section{Metodologia i rigor}

\subsection{Metodologia}
Per maximitzar la productivitat d'un equip de desenvolupadors, és important tenir una bona metodologia. Així, s'evita que la feina d'un membre de l'equip col·lideixi, endarrereixi o impedeixi la d'un altre. Per aquest motiu, les metodologies Àgils han sigut l'elecció per excel·lència pel desenvolupament d'aquest projecte, més concretament s'ha usat \textit{Scrum}.

Seguint la metodologia \textit{Scrum}, la feina s'ha organitzat de manera que es puguin realitzar \textit{Sprints}. Els \textit{Sprints} són iteracions de dues o tres setmanes durant les quals s'implementen funcionalitats noves que s'afegeixen al producte intentant mantenir-lo sempre usable. Els \textit{Sprints} es finalitzen amb una reunió on s'avalua el progrés i es decideix què implementar durant la següent iteració. Les tasques que s'han decidit implementar han intentat ser d'una durada igual o menor a la durada del \textit{Sprint}, per tant, la feina s'ha dividit en subtasques per arribar a aquesta quota.

A més a més de les reunions anteriorment mencionades, l'equip també s'ha reunit de manera diària (\textit{Daily Scrum} o \textit{Daily Standup}) i setmanal (\textit{Weeklys}). Aquestes reunions més breus han servit per mantenir als desenvolupadors actualitzats i col·laborant mútuament, poder detectar a temps qualsevol problema que pugui sorgir. Independentment d'aquestes reunions, l'equip ha estat comunicat mitjançant un programa de missatgeria instantània.

Per evitar errors al codi, s'ha utilitzat un mètode de desenvolupament conegut com a \textit{Test Driven Development} (TDD), que consisteix a convertir els requisits de programari en casos de prova abans de crear el mateix codi. D'aquesta manera, es crea una gran quantitat de proves al llarg del desenvolupament que verifiquen constantment que es compleixen tots els requisits, cosa que garanteix que el codi funcioni correctament.

A continuació es defineixen els diferents rols dins del marc de la metodologia \textit{Scrum} \cite{scrum} i quines persones hi han format part de cada rol en el projecte.

\subsubsection{Propietari del producte (Product Owner)}
El propietari del producte representa el negoci i és responsable d'assegurar que l'equip ofereix el màxim valor, cosa que requereix una relació de confiança amb l'equip de desenvolupament.

El propietari del producte ha de prioritzar la feina basant-se en diversos factors, incloses les necessitats del client i els requisits de les parts interessades (\textit{stakeholders}). Si entressin les prioritats en conflicte, podrien obstaculitzar l'eficàcia de l'equip. Les responsabilitats del \textit{Product Owner} són la gestió del backlog, la supervisió dels desplegaments i el maneig dels \textit{stakeholders}.

En general, el propietari del producte és fonamental per alinear l'equip amb els objectius empresarials i facilitar una comunicació i una col·laboració eficaces entre els \textit{stakeholders}.

En aquest projecte, el propietari del producte o \textit{Product Owner} ha estat \textbf{i2CAT}.


\subsubsection{Facilitador (Scrum Master)}
El facilitador és el paper responsable de garantir l'aplicació efectiva de les pràctiques \textit{Scrum}. Actuant com a líder servidor, facilita la comunicació i la col·laboració dins de l'equip. Se centren a fer visible el treball de l'equip, el foment d'una cultura d'aprenentatge i el foment d'un entorn alineat amb els valors de \textit{Scrum}.

Garanteix la transparència mitjançant la creació de mapes d'històries i l'actualització de la documentació; entrena l'equip en el repartiment del treball i millorar amb la revisió dels resultats; promou l'autoorganització encoratjant els membres de l'equip a provar nous enfocaments; i posa èmfasi en la importància dels valors \textit{Scrum} en la creació d'un ambient de confiança.

Dins de l'àmbit del projecte, el Facilitador o \textit{Scrum Master} ha sigut el \textbf{cap del projecte}.


\subsubsection{Desenvolupadors (Development Team)}
Contràriament a la percepció comuna, el terme ``desenvolupador'' abasta diverses funcions com a dissenyadors, escriptors i programadors, no només enginyers. Les responsabilitats principals dels desenvolupadors són l'autoorganització, el disseny, el desenvolupament, les proves i el desplegament. Tenen autonomia per prendre decisions, ja que l'autoorganització no consisteix a desafiar l'organització, sinó a capacitar els qui estan més a prop de la feina. 

Es comprometen a lliurar la feina dins d'un esprint i garanteix la transparència mitjançant reunions diàries de \textit{Scrum} (\textit{Daily Scrum}). Tot i que el \textit{Scrum Master} pot facilitar el \textit{Scrum} diari, en última instància és responsabilitat de l'equip de desenvolupament dirigir la reunió, fomentant la col·laboració i la millora contínua a la feina.

En aquest projecte, l'equip de desenvolupadors o \textit{Development Team} han sigut els \textbf{dos desenvolupadors júniors}.






\subsection{Eines}

\begin{itemize}
    \item \textbf{Git:} És una eina que és utilitzada per controlar i gestionar les versions del codi. També s'utilitza per compartir el codi amb el client.
    \item \textbf{Python:} llenguatge de programació per acomplir les tasques de \textit{Machine Learning} (ML) i consensuat amb l'empresa.
    \item \textbf{Hugging Face:} Principal font d'investigació sobre models i facilitadora d'eines per la seva prova i execució.
    \item \textbf{OTRS:} Sistema lliure que s'utilitza per assignar identificadors únics a sol·licituds de servei o informació. És el sistema utilitzat a la primera base de dades d'on s'extreuen els tiquets.
    \item \textbf{Elasticsearch:} Servidor que proveeix un motor de cerca de text complet, distribuït i amb una interfície web. És publicat com a codi obert i s'utilitza per a la segona base de dades on es guarden els tiquets.
    \item \textbf{Logstash:} És un \textit{pipeline} de processament de dades que transforma les dades abans de ser emmagatzemades a Elasticsearch. S'ha utilitzat per pseudoanonimitzar els camps abans d'emmagatzemar-los.
    \item \textbf{Models NLP\cite{Hugging-Face}}: Escollit després de fer l'estudi corresponent.
    \item \textbf{Slack:} El servei de missatgeria instantània usat per comunicar-se amb l'equip.
    \item \textbf{Google Meet:} Es fa servir per celebrar les reunions digitals.
\end{itemize}