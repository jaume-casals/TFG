\section{Metodologia i rigor}

\subsection{Metodologia}
Per maximitzar la productivitat d'un equip de desenvolupadors, és important tenir una bona metodologia. Així, s'evita que la feina d'un membre de l'equip col·lideixi, endarrereixi o impedeixi la d'un altre. Per aquest motiu, les metodologies Àgils seran l'elecció per excel·lència pel desenvolupament d'aquest projecte, més concretament s'utilitza \textit{Scrum}.

Seguint la metodologia \textit{Scrum}, la feina s'organitza de manera que es puguin realitzar \textit{Sprints}. Els \textit{Sprints} són iteracions de dues o tres setmanes durant les quals s'implementaran funcionalitats noves que s'afegeixen al producte intentant mantenir-lo sempre usable. Els \textit{Sprints} es finalitzen amb una reunió on s'avalua el progrés i es decideix què implementar durant la següent iteració. Les tasques que es decideixin implementar han de ser d'una durada igual o menor a la durada del \textit{Sprint}, per tant, la feina s'ha de dividir en subtasques per arribar a aquesta quota.

A més a més de les reunions anteriorment mencionades, l'equip també es reuneix de manera diària (\textit{dailys}) i setmanals (\textit{weeklys}). Aquestes reunions més breus serveixen per mantenir als desenvolupadors actualitzats i col·laborant mútuament, poden detectar a temps qualsevol problema que pugui sorgir. Independentment d'aquestes reunions, l'equip està comunicat mitjançant un programa de missatgeria instantània.

Per evitar errors al codi, s'utilitzarà un mètode de desenvolupament conegut com a \textit{Test Driven Development} (TDD), que consisteix a convertir els requisits de programari en casos de prova abans de crear el mateix codi. D'aquesta manera, es crea una gran quantitat de proves al llarg del desenvolupament que verifiquen constantment que es compleixen tots els requisits, cosa que garanteix que el codi funcioni correctament.

\subsection{Eines}

\begin{itemize}
    \item \textbf{Git:} És una eina que és utilitzada per controlar i gestionar les versions del codi. També s'utilitza per compartir el codi amb el client.
    \item \textbf{Python:} llenguatge de programació per acomplir les tasques de \textit{Machine Learning} (ML) i consensuat amb l'empresa.
    \item \textbf{Hugging Face:} Principal font d'investigació sobre models i facilitadora d'eines per la seva prova i execució.
    \item \textbf{OTRS:} Sistema lliure que s'utilitza per assignar identificadors únics a sol·licituds de servei o informació. És el sistema utilitzat a la primera base de dades d'on s'extreuen els tiquets.
    \item \textbf{Elasticsearch:} Servidor que proveeix un motor de cerca de text complet, distribuït i amb una interfície web. És publicat com a codi obert i s'utilitza per a la segona base de dades on es guarden els tiquets.
    \item \textbf{Models NLP\cite{BERT}\cite{RoBERTaA}\cite{GPT2}}: Escollit després de fer l'estudi corresponent.
    \item \textbf{Slack:} El servei de missatgeria instantània usat per comunicar-se amb l'equip.
    \item \textbf{Google Meet:} Es farà servir per celebrar les reunions digitals.
\end{itemize}