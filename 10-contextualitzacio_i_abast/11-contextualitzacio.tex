\section{Contextualització} \label{sec:context}


\subsection{Context}
En una era dominada per la digitalització, en la qual la tecnologia ha revolucionat innegablement les nostres vides i les operacions empresarials, ha sorgit un adversari formidable: una onada de ciberamenaces i frau digital. En endinsar-nos en l'àmbit de la ciberseguretat, ens enfrontem a la necessitat urgent d'adoptar mesures sòlides per contrarestar les ramificacions del frau digital, que van molt més enllà de l'àmbit virtual i s'infiltren en la nostra societat.

Aquesta revolució digital ha donat lloc a un panorama cada cop més desafiador per a la ciberseguretat. A mesura que més persones i empreses depenen de plataformes i serveis en línia, els riscos i costos dels ciberatacs i fraus es disparen a nivells sense precedents. Les estafes informàtiques el 2023 (que representen quasi el 90\% de tota la cibercriminalitat) presenten un increment més del 20\% sobre el mateix període (de gener a juny) del 2022. Per comprendre millor encara l'evolució de la cibercriminalitat, i el seu impacte sobre el conjunt de la criminalitat, les estafes informàtiques van representar la quantitat anual de 335.995 delictes el 2022 i 70.178 al 2016. Això implica que, en només sis anys, les estafes informàtiques l'any 2022 van créixer més del 370\% sobre les del 2016. \cite{criminalitat} Aquesta tendència es pot veure en els gràfics que mostren algunes estadístiques a la secció \ref{sec:stats_frau_digital}.

L'impacte d'aquestes ciberamenaces transcendeix les pèrdues financeres meres, impregnant les vides d'individus i organitzacions per igual. Les persones ja no són mers espectadors, sinó que es troben a primera línia d'una guerra cibernètica, en què la informació personal, abans considerada sagrada, s'ha convertit en un bé preuat per als actors maliciosos que aprofites totes les vulnerabilitats. Alhora, les empreses s'enfronten a una allau d'atacs que posen en perill no només la seva estabilitat financera, sinó que també erosionen la confiança i comprometen informació delicada.

El frau digital, més enllà de les implicacions financeres, és un repte social. Soscava la confiança dels consumidors i les empreses en l'entorn en línia, amb grans conseqüències a la privadesa, la seguretat i el benestar general. El robatori d'identitat, el \textit{phishing} i l'apropiació de comptes exposen informació personal i financera sensible, comprometent comptes i transaccions en línia. La reputació i la credibilitat d'individus i organitzacions estan en joc.

Una de les formes d'engany digital més freqüents són els ciberatacs dirigits al sector sanitari. Aquest àmbit és especialment susceptible de patir atacs de \textit{ransomware} (segrest de dades) que bloquegen dades i exigeixen un pagament a canvi del seu alliberament. Aquests atacs poden tenir greus conseqüències tant per als professionals sanitaris com per als pacients, com ara posar en perill la seguretat dels pacients, interrompre els serveis mèdics i comprometre informació confidencial.

Diversos factors contribueixen a fer que el sector sanitari sigui més vulnerable als ciberatacs que altres sectors. Un problema clau és la dependència de programari i sistemes antiquats dins de la sanitat. Aquests programes funcionen sovint amb sistemes operatius obsolets, que ja no reben manteniment ni actualitzacions dels seus fabricants. Aquests sistemes són susceptibles de ser vulnerats a través de falles conegudes que els permeten accedir als dispositius connectats. Un altre factor és el gran valor i importància de les dades de les organitzacions sanitàries que es conserven i manipulen. Aquestes dades inclouen historials mèdics, receptes i altra informació confidencial que els ciberdelinqüents i els competidors poden explotar per a activitats fraudulentes. La corrupció de dades per delictes cibernètics pot tenir greus conseqüències, com ara retards en el diagnòstic, el tractament i els errors de prescripció.

\subsection{Justificació}
A l'àmbit de la ciberseguretat contemporània, la gestió de tiquets d'incidents constitueix un component operatiu crític per a les agències dedicades a salvaguardar les infraestructures digitals. Reconeixent la importància primordial d'una gestió eficient dels tiquets d'incidents, una agència de ciberseguretat destacada s'ha embarcat en un projecte confidencial destinat a extreure i analitzar informació relativa als tiquets de ciberseguretat que reben i posteriorment resolen. Aquesta iniciativa, orquestrada per l'\textbf{Agència de ciberseguretat}, va requerir la contractació d'un intermediari, \textbf{i2CAT}, per facilitar els processos d'extracció i anàlisi. Alhora, \textbf{i2CAT} va confiar l'execució d'aquesta intrincada tasca a una altra entitat, \textbf{inLab FIB}, on treballa l'autor.

Dins d'aquest marc, l'objectiu general del projecte és la recuperació, anàlisi i emmagatzematge segur de la informació delicada relacionada amb aquests tiquets de ciberseguretat. S'ha posat especial èmfasi en respectar els matisos del sistema de gestió de tiquets emprat per l'\textbf{Agència}. La finalitat d'aquesta iniciativa ha sigut millorar la capacitat d'aquesta entitat per classificar eficaçment les possibles futures incidències.

La naturalesa intrínsecament sensible del projecte ha requerit un enfocament rigorós i altament confidencial, com subratllen els múltiples acords de confidencialitat (\textit{NDA}) que regeixen les interaccions entre totes les parts implicades: l'\textbf{Agència de ciberseguretat}, \textbf{i2CAT} i \textbf{inLab FIB}. En conseqüència, el projecte s'ha caracteritzat per unes mesures de seguretat estrictes de les dades per garantir que tota la informació relativa als tiquets romangui segura dins dels servidors de l'\textbf{Agència}.


\subsection{Problema a resoldre} \label{ssec:problema-resoldre}
El panorama actual de la ciberseguretat està marcat per una sèrie d'amenaces digitals en evolució constant, que requereixen la millora contínua de les mesures defensives i les estratègies de resposta. Un aspecte fonamental d'aquesta postura defensiva gira al voltant de la gestió eficient dels tiquets d'incidents de ciberseguretat. Aquests tiquets serveixen per documentar i fer un seguiment dels incidents notificats, siguin correus de \textit{phishing}, anomalies o programari maliciós. Un sistema de gestió de tiquets d'incidents ben estructurat és indispensable per permetre una ràpida resolució de les amenaces. Aquest projecte ha abordat un problema específic associat a aquesta faceta crucial de la gestió de la ciberseguretat.

L'\textbf{Agència} depèn en gran manera d'un sistema de gestió de tiquets d'incidents per gestionar i resoldre incidents de ciberseguretat. Tot i això, el sistema existent emprat per l'\textbf{Agència} ha mostrat certes deficiències que necessiten rectificació. Una d'aquestes, és l'absència d'un mecanisme per analitzar les dades contingudes als tiquets. Això ha plantejat reptes importants per a la capacitat de l'\textbf{Agència} d'obtenir informació pràctica a partir de les dades històriques dels tiquets i aplicar mesures proactives per frustrar les amenaces recurrents.

Per exemple, considerem un escenari on l'\textbf{Agència} s'ha trobat prèviament amb un sofisticat atac de \textit{phishing} que utilitzava un mètode d'atac novell. El tiquet d'incident associat a aquest atac conté informació molt valuosa sobre el modus operandi de l'atac, el punt d'origen de l'atac, les accions de mitigació de l'atac o els usuaris afectats. L'actual sistema d'incidents no permet l'extracció i posterior anàlisi sistemàtica d'aquesta informació valuosa. En conseqüència, quan torni a sorgir un mètode similar, la capacitat de l'\textbf{Agència} per accelerar-ne la resposta i mitigar els possibles danys es veu obstaculitzada per la manca d'informació històrica.

L'objectiu principal d'aquest projecte ha sigut dissenyar i implementar un sistema d'anàlisi i extracció de la informació de tiquets d'incidents que alleugi les deficiències existents. En el projecte també s'ha inclòs el desenvolupament d'una API que permeti utilitzar el sistema de manera senzilla. S'ha implementat també una funcionalitat que permet l'arxiu segur de les dades dins dels seus propis servidors per tal de proveir una anonimització de la informació més sensible. Es preveu que aquest sistema doti l'\textbf{Agència} de la capacitat d'extreure informació de les dades històriques dels tiquets, facilitant la ràpida detecció i la resposta a amenaces recurrents i el desenvolupament de mesures preventives.

Amb el compliment d'aquest objectiu, el projecte aspira a satisfer la bretxa existent entre la notificació i l'anàlisi d'incidents, permetent així a l'\textbf{Agència} aprofitar tot el potencial de les dades de tiquets d'incidents. Aquest sistema millorat de gestió de tiquets d'incidents garanteix que les dades crítiques romanguin accessibles, confidencials i en compliment dels protocols de seguretat. 

En essència, el projecte soluciona una deficiència de l'actual sistema de gestió d'incidents de l'\textbf{Agència}, facilitant l'extracció sistemàtica, l'anàlisi i l'emmagatzematge segur de les dades dels incidents, reforçant en darrer terme la capacitat de l'\textbf{Agència} per detectar, respondre i prevenir les amenaces a la ciberseguretat amb més eficàcia.


\subsection{Actors implicats}
Són actors totes aquelles parts que, o bé els seus interessos es poden veure afectats positivament o negativament pels resultats d'aquests, o bé estan implicades de forma directa en el projecte. Aquests són els següents:

\begin{itemize}
    \item \textbf{L'agència de ciberseguretat:} D'ara endavant, l'\textbf{Agència}, la principal part interessada i beneficiària del sistema d'anàlisi automàtica d'incidents és la mateixa agència de ciberseguretat. L'objectiu del sistema és millorar la seva eficiència operativa general, proporcionant-los una eina per a l'extracció, anàlisi i emmagatzematge segur de dades d'incidents. L'\textbf{Agència} utilitza aquest sistema com a eina vital per a una detecció, resposta i prevenció d'incidents més eficaç.
    \item \textbf{i2CAT:} És l'intermediari contractat per l'\textbf{Agència} per executar el projecte. És responsable que el sistema de gestió de tiquets d'incidents arribi a bon port. \textbf{i2CAT} és una part interessada en l'èxit del projecte i utilitzarà el sistema durant el desplegament per satisfer les necessitats de l'\textbf{Agència}. Es beneficia del compliment de les obligacions contractuals i, potencialment, de l'èxit del desplegament del sistema en altres projectes o contractes.
     \item \textbf{inLab FIB:} És el subcontractista contractat per \textbf{i2CAT} per implantar el sistema de gestió de tiquets d'incidències. Són els responsables directes del desenvolupament de la solució tècnica i de garantir-ne la funcionalitat. Els interessos d'\textbf{inLab FIB} resideixen a lliurar un producte funcional que satisfaci els requisits de l'\textbf{Agència}, així com complir les obligacions amb el soci contractual, \textbf{i2CAT}.
\end{itemize}


\subsection{Identificació de lleis i regulacions}
En la realització d'aquest projecte, ha sigut essencial considerar el marc legal i reglamentari que regeix el tractament de dades confidencials i l'execució de les obligacions contractuals. La base de la confidencialitat i el compliment legal d'aquest projecte són les lleis de \textit{LOPDGDD}, \textit{GDPR} i la signatura d'\textit{Acords de No Divulgació} (\textit{NDA}) entre les parts implicades, inclosa l'\textbf{Agència}, \textbf{i2CAT} i \textbf{inLab FIB}. 

La llei \textit{LOPDGDD} regula la protecció de dades personals i els drets digitals al context espanyol. Estableix principis com ara la necessitat d'obtenir consentiment per processar dades, la limitació en la recopilació de dades i l'obligació d'implementar mesures de seguretat.

El \textit{GDPR} és una regulació de la Unió Europea que harmonitza les lleis de protecció de dades a tots els estats membres. Proporciona un marc legal robust per al tractament de dades personals, amb èmfasi en el respecte a la privadesa i els drets individuals.

Els \textit{NDA} són fonamentals per restringir la difusió d'informació més enllà de les persones i entitats designades que participen directament al projecte. L'incompliment d'un acord de confidencialitat pot comportar conseqüències jurídiques, incloent-hi possibles litigis civils i danys i perjudicis. Per conseqüència, aquest projecte s'ha sotmès als convenis establerts per l'acord signat per tots els integrants d'aquest projecte, que conté la restricció de no compartir informació confidencial amb individus externs al projecte.

